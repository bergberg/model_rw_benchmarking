\documentclass[]{article}
\usepackage{lmodern}
\usepackage{amssymb,amsmath}
\usepackage{ifxetex,ifluatex}
\usepackage{fixltx2e} % provides \textsubscript
\ifnum 0\ifxetex 1\fi\ifluatex 1\fi=0 % if pdftex
  \usepackage[T1]{fontenc}
  \usepackage[utf8]{inputenc}
\else % if luatex or xelatex
  \ifxetex
    \usepackage{mathspec}
    \usepackage{xltxtra,xunicode}
  \else
    \usepackage{fontspec}
  \fi
  \defaultfontfeatures{Mapping=tex-text,Scale=MatchLowercase}
  \newcommand{\euro}{€}
\fi
% use upquote if available, for straight quotes in verbatim environments
\IfFileExists{upquote.sty}{\usepackage{upquote}}{}
% use microtype if available
\IfFileExists{microtype.sty}{\usepackage{microtype}}{}
\usepackage{graphicx}
\makeatletter
\def\maxwidth{\ifdim\Gin@nat@width>\linewidth\linewidth\else\Gin@nat@width\fi}
\def\maxheight{\ifdim\Gin@nat@height>\textheight\textheight\else\Gin@nat@height\fi}
\makeatother
% Scale images if necessary, so that they will not overflow the page
% margins by default, and it is still possible to overwrite the defaults
% using explicit options in \includegraphics[width, height, ...]{}
\setkeys{Gin}{width=\maxwidth,height=\maxheight,keepaspectratio}
\ifxetex
  \usepackage[setpagesize=false, % page size defined by xetex
              unicode=false, % unicode breaks when used with xetex
              xetex]{hyperref}
\else
  \usepackage[unicode=true]{hyperref}
\fi
\hypersetup{breaklinks=true,
            bookmarks=true,
            pdfauthor={P. J. van den Berg},
            pdftitle={Bayesian model benchmarking (working title)},
            colorlinks=true,
            citecolor=blue,
            urlcolor=blue,
            linkcolor=magenta,
            pdfborder={0 0 0}}
\urlstyle{same}  % don't use monospace font for urls
\setlength{\parindent}{0pt}
\setlength{\parskip}{6pt plus 2pt minus 1pt}
\setlength{\emergencystretch}{3em}  % prevent overfull lines
\setcounter{secnumdepth}{0}

\title{Bayesian model benchmarking (working title)}
\author{P. J. van den Berg}
\date{6 August, 2014}

\begin{document}
\maketitle

\subsection{Introduction}\label{introduction}

\subsection{Motivation}\label{motivation}

\subsection{Model description}\label{model-description}

We are given a dataset of ratings

\[\left[ x_{ij}\right]_{I_{ij}=1}\in\mathbb{R}, i=1\dots N,j=1\dots M, I_{ij} \in [0,1]\]

of $M$ counterparties by $N$ banks. The ratings $x$ are transformed to
be numbers on the real line and standardized; for instance, if given as
probability of default estimates $p\in[0,1]$, we transform these as
$x=\Phi^{-1}(p)$ and $x\to(x - \bar{x}) / \mathrm{sd}(x)$. Not all
counterparties are rated by all banks; we write this as an incidence
matrix $I_{ij}$ where $I_{ij}=1$ if a rating by bank $i$ for
counterparty $j$ exists, and $0$ otherwise.

Our model is particularly simple:

\[ x_{ij|I_{ij}=1}  \sim  \mathrm {Normal} (q_{j} + \mu_{i},\sigma_i ) \]
\[ I_{ij}  \sim \mathrm{Bernoulli}(p) \]

Each rating $x_{.j}$ is an estimate of the (unknown) `true' rating
(probability of default, loss given default, \ldots{}) $q_j$ of that
counterparty. We assume that each bank $i$ uses a model characterized by
a bias $\mu_i$ and error $\sigma_i$ which are the same for each rating
of that bank, and we assume that these are uncorrelated between banks.
We also assume whether a rating is present is independent of either bank
or counterparty.

As it is, this model suffers from an ?-way invariance; under the
simultaneous transformations

\[
\begin{matrix} q_j \to q_j+a \\ \mu_i\to\mu_i-a_{\{i,j\}_k} \end{matrix} 
\]

the resulting distribution of $x_{ij}$ will be the same. We remove this
collinearity by specifying a prior on the $q_j$ which breaks the
symmetry. The bias parameters represent the bias relative to the average
rating. We also remove any dependence on $x_{.j}$ where
$\sum_1^N x_{ij}=1$, i.e., counterparties for which only one rating is
available.

We wish to estimate the marginal posterior density for the $\mu_i$,

\[P(\mu|x,I,M,N)=\int\dots\int P(\mu|\tau,x,I,M,N)\mathrm{d}\tau_1 \dots \mathrm{d}\tau_M\]

We choose weakly informative conjugate joint priors for the $\mu_i$ and
$\tau_i$,

\[\mathrm{P}(\mu_i,\tau_i|\dots)=\mathrm{NormalGamma}(\mu_{0i},\nu_i,\alpha_i,\beta_i)\]

with

\[\begin{matrix} \mu_{0i}=0, i=1\dots N \\ \nu_i = \mathrm{large number}\end{matrix}\]

{[}\ldots{}{]}

\subsection{Data}\label{data}

As an example, we use data from the 2012 Dutch Hypothetical Portfolio
Exercise (HPE). These data contain PD and LGD predictions for 342
corporate counterparties by 7 banks and rating agencies. We exclude PD
values equal to 1, set PD/LGD values equal to 1(0) to $\Phi(+(-)8.1259)$
(corresponding to the largest (smallest) representable double precision
float) and normalize as

\[ \begin{matrix} \mathrm{R} \to x = (\Phi^{-1}(\mathrm{R}) - \mathrm{mean}(x')) / \mathrm{sd}(x) \\ \mathrm{with} \\ x' = \Phi^{-1}(\mathrm{R})  \end{matrix} \]

where $\mathrm{R}$ is either the PD or LGD. Figures \textbf{???} show
the resulting distributions for the biases $\mathbf{mu}$

\subsection{Results}\label{results}

\begin{figure}[htbp]
\centering
\includegraphics{C:/Users/rn8089/Desktop/LGD_dutch_mortgages.png}
\caption{A voyage to the moon\label{fig:lalune}}
\end{figure}

See \hyperref[fig:lalune]{figure \ref{fig:lalune}}.

\section{Appendix: Stan
implementation}\label{appendix-stan-implementation}

\begin{verbatim}
data {
    int<lower=0> N; // number of models      
    int<lower=0> M; // number of subjects

    real x[M,N]; // standardized model estimates
    int<lower=0,upper=1>I[M,N]; // is rating (i,j) present or not
}

parameters {
    real q[M]; // true values
    real y[M,N]; // unobserved ratings
    real mu[N]; // model relative bias
    real<lower=0>tau[N]; // model precision
    real<lower=0,upper=1>p; // probability of having a rating
    real<lower=0>a; // 
    real<lower=0>b; // parameters of the prior to p

}

transformed parameters {    
    real<lower=0>sigma[N];
    for (j in 1:N)
        sigma[j] <- pow(tau[j], -0.5);
}

model {
    real X[M,N];  // combination of observed and unobserved ratings

    for (i in 1:M)
        for (j in 1:N)
            if (I[i,j]==1)
                X[i,j] <- x[i,j];
            else
                X[i,j] <- y[i,j];

    for (i in 1:M)
        for (j in 1:N) {
            increment_log_prob(normal_log(X[i,j],q[i] + mu[j],sigma[j])); //X[i,j] ~ normal(q[i] + mu[j], sigma[j]);    
            I[i,j] ~ bernoulli(p);
        }

    for (j in 1:N) {
        mu[j] ~ cauchy(0,10);
        tau[j] ~ cauchy(0,10);
    }

    for (i in 1:M)
        q[i] ~ cauchy(0,10);

    p ~ beta(a,b);
}
\end{verbatim}

\section{Appendix: Diagnostic output}\label{appendix-diagnostic-output}

The following diagnostic results serve to ascertain the convergence of
the MCMC sampling procedure.

Stan {[}@stan-software:2014{]}

\end{document}
